%%%%%%%%%%%%%%
%  COSTANTI  %
%%%%%%%%%%%%%%
%COMANDI DA UTILIZZARE PER TUTTI I DOCUMENTI
% In questa prima parte vanno definite le 'costanti' utilizzate da due o più documenti.

%COMANDI TABELLE
\newcommand{\rowcolorhead}{\rowcolor[HTML]{00ACC1}} %intestazione 
\newcommand{\rowcolorlight}{\rowcolor[HTML]{E0f7fa}} %righe chiare/dispari
\newcommand{\rowcolordark}{\rowcolor[HTML]{80DEEA}} %righe scure/pari
\newcommand{\colorhead}{\color[HTML]{FFFFFF}} %testo intestazione
\newcommand{\colorbody}{\color[HTML]{000000}} %testo righe


%%%%%%%%%%%%%%
%  FUNZIONI  %
%%%%%%%%%%%%%%

% Serve a dare la giusta formattazione al codice inline
\newcommand{\code}[1]{\flextt{\texttt{#1}}}

% Genera automaticamente la pagina di copertina
\newcommand{\makeFrontPage}{
  % Declare new geometry for the title page only.
  \newgeometry{top=4cm}
  
  \begin{titlepage}
  \begin{center}

\begin{center}
  \centerline{\includegraphics[scale=0.24]{StyleLatex/logo_home.png}}
\end{center}
  
  \vspace{1cm}

  \begin{Huge}
  \textbf{\DocTitolo{}} \\
  \end{Huge}

  \vspace{9pt}  
  
  \begin{tabular}{l|l}
  	\textbf{Nome Gruppo} & WalkAndBuy \\
  	\textbf{Componenti} & Canovese Marco 1094135\\
  	& Zanellato Federico 1094134\\

  \end{tabular}

  \vspace{30pt}
  

  \end{center}
  \end{titlepage}
  
  % Ends the declared geometry for the titlepage
  \restoregeometry
}
\input{StyleLatex/layout.tex}
%%%%%%%%%%%%%%
%  COSTANTI  %
%%%%%%%%%%%%%%

% In questa prima parte vanno definite le 'costanti' utilizzate soltanto da questo documento.
% Devono iniziare con una lettera maiuscola per distinguersi dalle funzioni.

\newcommand{\DocTitolo}{Errori HTML rilevati nel progetto}

\newcommand{\DocRedazione}{Walk and Buy}

\newcommand{\DocAnno}{Anno 2017/2018}

\title{\textbf{Relazione progetto TecWeb}}
\author{WalkAndBuy}

\date{11 Settembre 2018}

\begin{document}

%\maketitle

\makeFrontPage

\tableofcontents

\newpage

\section{Abstract}

Lo scopo del progetto è quello di creare un e-commerce gestito dall’Associazione Agricoltori del Camposampierese, la quale a fronte della inarrestabile concorrenza dei grandi colossi del commercio online ha deciso di favorire un servizio dedicato agli agricoltori e commercianti del territorio.
Il sito web permette infatti alle aziende del territorio, previo accordo con l’Associazione, di promuovere e vendere i propri prodotti online. 
Il servizio offerto è limitati ai cittadini residenti nei comuni ove l’Associazione ha dei soci attivi. 
I prodotti vendibili devono essere afferenti ad una delle macro categorie previste, ovvero 
\begin{itemize}
	\item Frutta e Verdura;
	\item Carne e Pesce;
	\item Formaggi;
	\item Conserve e Uova;
	\item Vino e Bevande.
\end{itemize}
Ogni articolo viene presentato mediante:
\begin{itemize}
	\item Titolo;
	\item sintetica descrizione;
	\item un’immagine rappresentativa;
	\item la quantità venduta;
	\item il prezzo di rivendita al netto di eventuali sconti promozionali applicati.
\end{itemize}

I commercianti abilitati possono modificare gli articoli da essi pubblicati.\\
Gli utenti una volta autenticati possono consultare il catalogo online, aggiungere e rimuovere articoli dal carrello e, una volta selezionati tutti i prodotti di loro interesse, concludere l’ordine. \\
Hanno inoltre a disposizione una pagina in cui possono visualizzare i singoli ordini effettuati.


\subsection{Utenti destinatari}

Il sito è destinato a qualsiasi utente 

\subsection{Gestione dei dati}


\section{Fasi di progettazione}

Lo sviluppo del progetto si è suddiviso in varie fasi, raggruppabili nelle seguenti categorie:

\subsection{Analisi dei requisiti}


\subsection{Realizzazione ed implementazione delle funzionalità}

\subsection{Fase di test}

Ogni funzionalità implementata nel sito web è stata oggetto di numerosi test per verificarne il suo effettivo funzionamento. Gli strumenti utilizzati sono stati \textbf{W3C Markup Validation Service} (per validazione XHTML), \textbf{W3C CSS Validator} (per validazione CSS), \textbf{browserstack.com} e plugin per Chrome \textbf{IE-tab} (per test su diversi browser).

\section{Struttura del progetto}


\section{Database}


\section{Struttura sito}


\section{Adattabilità ed usabilità del sito}


\section{Compatibilità browser}


\end{document}