%\documentclass[a4paper, oneside, openany]{book}
\documentclass[10pt,a4paper]{extarticle}

%**************************************************************
% Importazione package
%************************************************************** 
\usepackage{graphbox}
% permette di modificare i margini
%\usepackage[top=3.1cm, bottom=3.1cm, left=2.2cm, right=2.2cm]{geometry}
%\usepackage[a4paper]{../../common/style/geometry} %TODO
\usepackage[a4paper, top=2cm, bottom=2.8cm]{geometry}
\usepackage{booktabs} %lista item nelle tabelle
% specifica con quale codifica bisogna leggere i file
\usepackage[utf8]{inputenc}

%indice con i puntini
\usepackage{tocloft}
\renewcommand\cftsecleader{\cftdotfill{\cftdotsep}}



% necessario per risolvere il problema del carattere invisibile per l'a capo
\DeclareUnicodeCharacter{00A0}{ } %?

% per scrivere in italiano e in inglese;
% l'ultima lingua (l'italiano) risulta predefinita
\usepackage[english, italian]{babel}

% imposta lo stile italiano per i paragrafi
\usepackage{parskip}

% fornice elenchi numerati in ordine inverso
\usepackage{etaremune}

\usepackage{caption}

% comandi per l'appendice
\usepackage{appendix}
%\renewcommand\appendixtocname{Appendici} %We really need it?

% import euro symbol
\usepackage{eurosym}

% numera anche i sottoparagrafi
\setcounter{secnumdepth}{5}

% elenca anche i sottoparagrafi nell'indice
\setcounter{tocdepth}{5}

% permetti di definire dei colori
\usepackage[usenames,dvipsnames]{color}

% permette di usare il comando "paragraph" come subsubsubsection!
\usepackage{titlesec}

% permette di inserire le immagini/tabelle esattamente dove viene usato il
% comando \begin{figure}[H] ... \end{figure}
% evitando che venga spostato in automatico
\usepackage{float}

% permette l'inserimento di url e di altri tipi di collegamento
\usepackage[colorlinks=true]{hyperref}

\hypersetup{
    colorlinks=true, % false: boxed links; true: colored links
    citecolor=black,
    filecolor=black,
    linkcolor=black, % color of internal links
    urlcolor=Maroon  % color of external links
}

% permette al comando \url{...} di andare a capo a metà di un link
\usepackage{breakurl}

% immagini
\usepackage{graphicx}

% permette di riferirsi all'ultima pagina con \pageref{LastPage}
\usepackage{lastpage}

% tabelle su più pagine
\usepackage{longtable}

% per avere dei comandi in più da poter usare sulle tabelle
\usepackage{booktabs}

% tabelle con il campo X per riempire lo spazio rimanente sulla riga
%\usepackage{tabularx}

% TABELLE 
% tabelle su più pagine
\usepackage{longtable}

% per avere dei comandi in più da poter usare sulle tabelle
\usepackage{booktabs}

% multirow per tabelle
\usepackage{multirow}

% definisci un nuovo tipo di colonna C che permette di andare a capo con \newline
% e centrata
\usepackage{array}
\usepackage{ragged2e}
%\newcolumntype{P}[1]{>{\RaggedRight\hspace{0pt}}p{#1}}
\newcolumntype{C}[1]{>{\centering\let\newline\\\arraybackslash\hspace{0pt}}m{#1}}

% colore di sfondo per le celle
\usepackage[usenames,dvipsnames,svgnames,table]{xcolor}

% tabelle con il campo X per riempire lo spazio rimanente sulla riga
\usepackage{tabularx}


% personalizza l'intestazione e piè di pagina
\usepackage{fancyhdr}

% permette di inserire caratteri speciali
\usepackage{textcomp}

% permette di aggiustare i margini e centrare tabelle e figure
\usepackage{changepage}

%Permette di includere i grafici a barre
%IMPORTANTE: deve essere caricato prima di /pgfgantt altrimenti causa conflitto
%\usepackage{pgfplots}

% permette di includere i diagrammi Gantt
% su Ubuntu non si può installare il pacchetto, deve essere in modello/
%\usepackage{../../modello/pgfgantt}

% permette di includere i grafici a torta
%\usepackage{../../modello/pgf-pie}

% necessario per pgf-pie
\usepackage{tikz}

% permette i path delle immagini con gli spazi
\usepackage{grffile}

% ruota le immagini
\usepackage{rotating}

% permetti di calcolare le larghezze facendo calcoli
\usepackage{calc}


\fancypagestyle{plain}{
	% cancella tutti i campi di intestazione e piè di pagina
	\fancyhf{}

	\lfoot{
		\DocTitolo{} \ - \textit{\DocRedazione{} \ -  \DocAnno{}}
	}
	\rfoot{Pagina \thepage{} di \pageref{LastPage}}

	% Visualizza una linea orizzontale in cima e in fondo alla pagina
	\renewcommand{\headrulewidth}{0pt}	
	\renewcommand{\footrulewidth}{0.3pt}
}
\pagestyle{plain}

% allarga l'header a tutta la pagina
%\fancyhfoffset[L]{\oddsidemargin + \hoffset + 1in}
%\fancyhfoffset[R]{\evensidemargin + \marginparwidth - \marginparsep}

% Per inserire del codice sorgente formattato

\usepackage{listings}

\lstset{
  extendedchars=true,          % lets you use non-ASCII characters
  inputencoding=utf8,   % converte i caratteri utf8 in latin1, richiede 
  %\usepackage{listingsutf8} anzichè listings
  basicstyle=\ttfamily,        % the size of the fonts that are used for the 
  %code
  breakatwhitespace=false,     % sets if automatic breaks should only happen at 
  %whitespace
  breaklines=true,             % sets automatic line breaking
  captionpos=t,                % sets the caption-position to top
  commentstyle=\color{mygreen},   % comment style
  frame=none,               % adds a frame around the code
  keepspaces=true,            % keeps spaces in text, useful for keeping 
  %indentation of code (possibly needs columns=flexible)
  keywordstyle=\bfseries,     % keyword style
  numbers=none,               % where to put the line-numbers; possible values 
  %are (none, left, right)
  numbersep=5pt,              % how far the line-numbers are from the code
  numberstyle=\color{mygray}, % the style that is used for the line-numbers
  rulecolor=\color{black},    % if not set, the frame-color may be changed on 
  %line-breaks within not-black text (e.g. comments (green here))
  showspaces=false,           % show spaces everywhere adding particular 
  %underscores; it overrides 'showstringspaces'
  showstringspaces=false,     % underline spaces within strings only
  showtabs=false,             % show tabs within strings adding particular 
  %underscores
  stepnumber=5,               % the step between two line-numbers. If it's 1, 
  %each line will be numbered
  stringstyle=\color{red},    % string literal style
  tabsize=4,                  % sets default tabsize
  firstnumber=1      % visualizza i numeri dalla prima linea
}

% Permetti di utilizzare il grassetto per i caratteri Typewriter (per es. il 
%font di \code{...} e \file{...})
\usepackage[T1]{fontenc}
\usepackage{lmodern}


%package added
\usepackage{amsmath}

\usepackage{amsfonts}
